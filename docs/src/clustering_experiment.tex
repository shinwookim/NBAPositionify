\documentclass[./main.tex]{subfiles}
\begin{document}
\subsection{Clustering}
\subsubsection{WCSS Analysis}
\begin{wrapfigure}{r}{5.5cm}
    \centering
    \includegraphics[width=0.35\textwidth]{photos/wcss_graph.png}
    \caption{A graph representing the Within Cluster Sum-of-Squares (WCSS) of 1 through 5 clusters from K-Means clustering.}
    \label{fig:wcss_graph}
\end{wrapfigure}
Before we performed clustering, our first experiment was to determine the optimal number of clusters to use based on WCSS. To do so, we ran the $k$-means clustering algorithm for all number of clusters between $1$ and $5$, and calculated WCSS and observed the results. We chose $5$ as the upper-bound for our clusters, since there are 5 distinct basketball positions. Our intuition was that selecting a number of clusters greater than $5$ would surely overfit the data.

From this experiment, we determined that 5 clusters best fits the data. 3 and 4 clusters appear to be similar in terms of WCSS, indicating that we might be able to group positions into 3 or 4 larger groups instead of the five traditional positional categories. However, for the purposes of attempting to differentiate players into the traditional 5 positions, this is the number of clusters we moved forward with in future experiments.

\subsubsection{K-Means}
\begin{figure}[h]
\centering
\begin{subfigure}{.4\textwidth}
    \includegraphics[width=\linewidth]{photos/kmeans_height_weight.png}
\end{subfigure}
\begin{subfigure}{.4\textwidth}
    \includegraphics[width=\linewidth]{photos/kmeans_ast_trb.png}
\end{subfigure}
\caption{Visual representation of k-means clusters based on the attributes clustering was performed on.}
\label{fig:kmeans_analysis}
\end{figure}
Since we already performed k-means clustering with 5 clusters as part of the WCSS experiment, we examined these clusters for further analysis. We constructed a visual representation of the positions of players within each cluster, as well as visualizing the relationship between the variables used for clustering and what cluster each player ended up in.

Each cluster contained mostly one specific position, except for the third cluster, which contained 49\% small forwards and 38\% power forwards. Clusters 2 and 4 both contained a majority of point guards, and no clusters contained a majority of centers. The clusters were relatively pure to one position each, but these specific cases meant that there were gaps left to be desired in using clustering as a way of grouping players together by position.

The positive to the k-means approach is, as seen in figure \ref{fig:kmeans_analysis}, there is a clear hierarchy when it comes to clusters and relevant statistics. As height and weight increase, the cluster that the data points are grouped in changes linearly. Similarly, high rebound percentage and low assist percentage are grouped in different clusters than high assist percentage and low rebound percentage.

\subsubsection{Agglomerative}



\begin{figure}[h]
    \centering
    \subcaptionbox{}{\includegraphics[width=0.3\linewidth]{photos/agg_c0.png}}
    \subcaptionbox{}{\includegraphics[width=0.3\linewidth]{photos/agg_c1.png}}
    \subcaptionbox{}{\includegraphics[width=0.3\linewidth]{photos/agg_c2.png}}
    \subcaptionbox{}{\includegraphics[width=0.3\linewidth]{photos/agg_c3.png}}
    \subcaptionbox{}{\includegraphics[width=0.3\linewidth]{photos/agg_c4.png}}
    \caption{The distribution of positions generated by the five clusters using agglomerative clustering.}
    \label{fig:agglomerative_analysis}
\end{figure}


Similar to k-means, we ran the agglomerative clustering algorithm with 5 clusters on the normalized dataset, and generated the same visualizations as k-means. We found that the relationship between the clusters and positions were more ``pure'' when creating the same visualizations as were created with the k-means clusters. As seen in Figure \ref{fig:agglomerative_analysis}, each of the clusters has above 50\% of records which are associated with one of the five specific positions, and each cluster's position which has the majority of records is unique as compared to the other clusters. That is to say, there is 1 majority PG cluster, 1 majority SG cluster, etc. In other words, it is more likely for a player to be grouped with other players at their true position in agglomerative clustering than k-means.




\subsection{Outliers}
To determine major outliers in our clustering, we used visual observation of the clusters—specifically the position which appeared the least in each cluster. Below, we detail some notable players who appeared in a cluster which was not associated with their primary listed position.

\textbf{Kevin Durant, 2016 and 2017. Clustered with C, but true position was SF.} Durant was drafted out of college in 2008 as a shooting guard, and over time as he grew taller and stronger transitioned to Small Forward and Power Forward role. In 2016 and 2017, the two seasons in question, Durant averaged 8.2 and 8.3 rebounds per game, respectively. In 2016, this was good for 25th in the league in this category. He is a good shooter, but also does not produce many assists. His height, high rebounds and low assists gives solid evidence to him being misclassified in this case, and his elite scoring ability and athleticism, not taken into account by our clustering model, might be a reason why his true position is instead a small forward.

\textbf{Jalen Rose, 1995. Clustered with PG, but true position was SF.} Rose had the build of a forward, as he is the tallest and biggest player out of the 6 players to be misclassified as a PG while being an SF. However, his 33.3\% assist percentage in his 1995 rookie year is what contributes to this misclassification. Rose was known to be a relatively good passer for someone in his position, and this is likely the main contributor to him being in the PG cluster. Later on in his career, Rose would go on to play some PG and SG from time to time, bringing further merit and reason behind this clustering.

\textbf{Dwyane Wade, 2004, Clustered with SF, but true position was PG.} In his prime, Wade was one of the best shooting guards in the game; a well-rounded scorer who had large enough of a build to be potentially confused with a small forward. The rookie year in which he was clustered with SFs was the only year in which his true position was not a shooting guard, and shooting guards and small forwards can often be interchanged for one another - especially if a shooting guard has a larger build, or can be considered more of a ``slasher'' (a perimeter player with the ability to drive to the basket).

\end{document}